\documentclass[a4paper, 11pt]{article}
\usepackage{comment} % enables the use of multi-line comments (\ifx \fi) 
\usepackage{fullpage} % changes the margin
\usepackage{amsmath}
\usepackage[makeroom]{cancel}

\usepackage{graphicx}
\usepackage[]{textcomp}

\usepackage{geometry}
 \geometry{
 a4paper,
 left=15mm,
 right=15mm,
 top=15mm,
 }

\makeatletter
\def\blfootnote{\xdef\@thefnmark{}\@footnotetext}
\makeatother

\begin{document}

\noindent \includegraphics[scale=0.2]{figures/logo.png}\\
\large\textbf{Projet de semestre - C\raisebox{.5\height}{\scalebox{.5}{++}} orienté objet} \hfill \textbf{Rafael RIBER  - SCIPER: 296142} \\
\large\textbf{Dodgeball} \hfill \hfill \textbf{Valentin RIAT - SCIPER: 289121}\\
\normalsize COM-112(a) \hfill \textbf{EL-BA2 - Semestre de printemps 2018-2019}\\

\section{Architecture logicielle et description de l’implémentation:}

\subsection{Structuration des données}

L'ensemble des joueurs est stocké dans un \texttt{vector} appelé \texttt{players}, attribut de la classe simulation. Les balles sont stockées de manière identique dans un \texttt{vector} appelé \texttt{balls}.

La carte est est stockée sous forme d'un vector à deux dimensions, où les cases ou un obstacle est présent sont représentées par '1', et les cases libres par '0'.

\section{Méthodologie et conclusion:}

\end{document}